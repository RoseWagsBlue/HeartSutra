%    printable flashcards for Wanicani kanji
%    uses the flacards class because it is easier than the more common flashcards class

\documentclass[letterpaper,frontgrid]{flacards}
%%\usepackage{color}
\usepackage{xeCJK}
\usepackage{fontspec}
\usepackage{anyfontsize}
\usepackage{hyperref}

\fboxsep=12pt
\setcounter{cardno}{0}

\hypersetup {
 pdfauthor= Rose DiFonzo,
 pdftitle={heart sutra flashcards},
 pdfsubject={kanji flashcards},
 pdfkeywords={kanji, japan, wanikani, heart sutra, buddha, buddhism},
 }
 
%%\renewcommand{\frfoot}{\footnotesize\thecardno\hskip3pt\smallskip} % change layout for left foot on front
%%\renewcommand{\brfoot}{\footnotesize\thecardno\hskip3pt\smallskip} % change layout for left foot on front
% You can define head and foot elements using \flhead, \fchead and \frhead for right, center and right head on frontside,
% \flfoot, \fcfoot, \flfoot for footline on frontside. Analogical use \blhead, \bchead, \brhead, \blfoot, \bcfoot and
% \frfoot for backside.
%\renewcommand{\cardtextstyleb}{\flushright}
%%\renewcommand{\frfoot}{\hskip3pt\footnotesize wanikani.com\smallskip}
\renewcommand{\flfoot}{\hskip3pt\footnotesize\thecardno\smallskip}
\renewcommand{\blfoot}{\footnotesize\thecardno}
\renewcommand{\frfoot}{}
\renewcommand{\brfoot}{}

\renewcommand{\cardtextstylef}{\fontsize{60}{60}\selectfont}
\renewcommand{\cardtextstyleb}{\Large}
\setCJKmainfont{KanjiStrokeOrders}
%% \setCJKmainfont{Meiryo}
 
\begin{document}
\pagesetup{4}{8} 

\card{佛}{ぶっ}
\card{説}{せつ}
\card{摩}{ま}
\card{訶}{か}
\card{般}{はん}
\card{若}{にゃ}
\card{波}{は}
\card{羅}{ら}
\card{蜜}{みっ}
\card{多}{た}
\card{心}{しん}
\card{経}{ぎょう}
\card{観}{かん}
\card{自}{じ}
\card{在}{ざい}
\card{菩}{ぼ}
\card{薩}{さつ}
\card{行}{ぎょう}
\card{深}{じん}
\card{般}{はん}
\card{若}{にゃ}
\card{波}{は}
\card{羅}{ら}
\card{蜜}{みっ}
\card{多}{た}
\card{時}{じ}
\card{照}{しょう}
\card{見}{けん}
\card{五}{ご}
\card{蘊}{うん}
\card{皆}{かい}
\card{空}{くう}
\card{度}{ど}
\card{一}{いっ}
\card{切}{さい}
\card{苦}{く}
\card{厄}{やく}
\card{舎}{しゃ}
\card{利}{り}
\card{子}{し}
\card{色}{しき}
\card{不}{ふ}
\card{異}{い}
\card{空}{くう}
\card{空}{くう}
\card{不}{ふ}
\card{異}{い}
\card{色}{しき}
\card{色}{しき}
\card{即}{そく}
\card{是}{ぜ}
\card{空}{くう}
\card{空}{くう}
\card{即}{そく}
\card{是}{ぜ}
\card{色}{しき}
\card{受}{じゅ}
\card{想}{そう}
\card{行}{ぎょう}
\card{識}{しき}
\card{亦}{やく}
\card{復}{ぶ}
\card{如}{にょ}
\card{是}{ぜ}
\card{舎}{しゃ}
\card{利}{り}
\card{子}{し}
\card{是}{ぜ}
\card{諸}{しょ}
\card{法}{ほう}
\card{空}{くう}
\card{相}{そう}
\card{不}{ふ}
\card{生}{しょう}
\card{不}{ふ}
\card{滅}{めつ}
\card{不}{ふ}
\card{垢}{く}
\card{不}{ふ}
\card{浄}{じょう}
\card{不}{ふ}
\card{増}{ぞう}
\card{不}{ふ}
\card{減}{げん}
\card{是}{ぜ}
\card{故}{こ}
\card{空}{くう}
\card{中}{ちゅう}
\card{無}{む}
\card{色}{しき}
\card{無}{む}
\card{受}{じゅ}
\card{想}{そう}
\card{行}{じょう}
\card{識}{しき}
\card{無}{む}
\card{眼}{げん}
\card{耳}{に}
\card{鼻}{び}
\card{舌}{ぜっ}
\card{身}{しん}
\card{意}{い}
\card{無}{む}
\card{色}{しき}
\card{聲}{しょう}
\card{香}{こう}
\card{味}{み}
\card{触}{そく}
\card{法}{ほう}
\card{無}{む}
\card{眼}{げん}
\card{界}{かい}
\card{乃}{ない}
\card{至}{し}
\card{無}{む}
\card{意}{い}
\card{識}{しき}
\card{界}{かい}
\card{無}{む}
\card{無}{む}
\card{明}{みょう}
\card{亦}{やく}
\card{無}{む}
\card{無}{む}
\card{明}{みょう}
\card{盡}{じん}
\card{乃}{ない}
\card{至}{し}
\card{無}{む}
\card{老}{ろう}
\card{死}{し}
\card{亦}{やく}
\card{無}{む}
\card{老}{ろう}
\card{死}{し}
\card{盡}{じん}
\card{無}{む}
\card{苦}{く}
\card{集}{しゅう}
\card{滅}{めつ}
\card{道}{どう}
\card{無}{む}
\card{智}{ち}
\card{亦}{やく}
\card{無}{む}
\card{得}{とく}
\card{以}{い}
\card{無}{む}
\card{所}{しょう}
\card{得}{とく}
\card{故}{こ}
\card{菩}{ぼ}
\card{提}{だい}
\card{薩}{さっ}
\card{埵}{た}
\card{依}{え}
\card{般}{はん}
\card{若}{にゃ}
\card{波}{は}
\card{羅}{ら}
\card{蜜}{みっ}
\card{多}{た}
\card{故}{こ}
\card{心}{しん}
\card{無}{む}
\card{罣}{けい}
\card{礙}{げ}
\card{無}{む}
\card{罣}{けい}
\card{礙}{げ}
\card{故}{こ}
\card{無}{む}
\card{有}{う}
\card{恐}{く}
\card{怖}{ふ}
\card{遠}{おん}
\card{離}{り}
\card{一}{いっ}
\card{切}{さい}
\card{顚}{てん}
\card{倒}{どう}
\card{夢}{む}
\card{想}{そう}
\card{究}{く}
\card{竟}{きょう}
\card{涅}{ね}
\card{槃}{はん}
\card{三}{さん}
\card{世}{ぜ}
\card{諸}{しょ}
\card{佛}{ぶつ}
\card{依}{え}
\card{般}{はん}
\card{若}{にゃ}
\card{波}{は}
\card{羅}{ら}
\card{蜜}{みっ}
\card{多}{た}
\card{故}{こ}
\card{得}{とく}
\card{阿}{あの}
\card{耨}{く}
\card{多}{た}
\card{羅}{ら}
\card{三}{さん}
\card{藐}{みゃ}
\card{三}{さん}
\card{菩}{ぼ}
\card{提}{だい}
\card{故}{こ}
\card{知}{ち}
\card{般}{はん}
\card{若}{にゃ}
\card{波}{は}
\card{羅}{ら}
\card{蜜}{みっ}
\card{多}{た}
\card{是}{ぜ}
\card{大}{だい}
\card{神}{しん}
\card{咒}{しゅ}
\card{是}{ぜ}
\card{大}{だい}
\card{明}{みょう}
\card{咒}{しゅ}
\card{是}{ぜ}
\card{無}{む}
\card{上}{じょう}
\card{咒}{しゅ}
\card{是}{ぜ}
\card{無}{む}
\card{等}{とう}
\card{等}{どう}
\card{咒}{しゅ}
\card{能}{のう}
\card{除}{じょ}
\card{一}{いっ}
\card{切}{さい}
\card{苦}{く}
\card{真}{しん}
\card{實}{じつ}
\card{不}{ふ}
\card{虚}{こ}
\card{故}{こ}
\card{説}{せつ}
\card{般}{はん}
\card{若}{にゃ}
\card{波}{は}
\card{羅}{ら}
\card{蜜}{みっ}
\card{多}{た}
\card{咒}{しゅ}
\card{即}{そく}
\card{説}{せつ}
\card{咒}{しゅ}
\card{曰}{わつ}
\card{羯}{ぎゃ}
\card{諦}{てい}
\card{羯}{ぎゃ}
\card{諦}{てい}
\card{波}{は}
\card{羅}{ら}
\card{羯}{ぎゃ}
\card{諦}{てい}
\card{波}{は}
\card{羅}{ら}
\card{僧}{そう}
\card{羯}{ぎゃ}
\card{諦}{てい}
\card{菩}{ぼ}
\card{提}{じ}
\card{薩}{そ}
\card{婆}{わ}
\card{訶}{か}
\card{般}{はん}
\card{若}{にゃ}
\card{心}{しん}
\card{經}{ぎょう}

\end{document} 
